\documentclass[12pt, a4paper, twoside, romanian]{teza-etti}
\setcounter{secnumdepth}{3}
\setcounter{tocdepth}{3}
\usepackage{array}
\usepackage{babel}
\usepackage{color}
\usepackage{graphicx}
\usepackage{markdown}
\usepackage{parskip}
\usepackage[
	bookmarks,
	bookmarksopen=true,
	pdftitle={Teza},
	linktocpage]{hyperref}

%%%%%%%%%%%%%%%%%%
%%%%%%%%%%%%%%%%%%
% cod colorat
% schimbați culorile și formatarea mai jos

\usepackage{listings}
\usepackage{color}

\definecolor{codegreen}{rgb}{0,0.6,0}
\definecolor{codegray}{rgb}{0.5,0.5,0.5}
\definecolor{codepurple}{rgb}{0.58,0,0.82}
\definecolor{backcolour}{rgb}{0.95,0.95,0.92}

\lstdefinestyle{mystyle}{
	language=Python,
	backgroundcolor=\color{backcolour},
	commentstyle=\color{codegreen},
	keywordstyle=\color{blue},
	numberstyle=\tiny\color{codegray},
	stringstyle=\color{codepurple},
	basicstyle=\ttfamily\scriptsize,
	breakatwhitespace=false,
	breaklines=true,
	captionpos=b,
	keepspaces=true,
	numbers=left,
	numbersep=5pt,
	showspaces=false,
	showstringspaces=false,
	showtabs=true,
	tabsize=2
}
\lstset{style=mystyle}
\renewcommand{\lstlistingname}{}
\renewcommand{\lstlistlistingname}{List of \lstlistingname s}% List of Listings -> List of Algorithms

\makeatletter
\providecommand*{\toclevel@unnumberedsection}{0}
\makeatother

\author{Popescu Ervin-Adrian}
\title{Aplicație de detecție și identificare a semnelor de circulație}
\tiplucrare{Diplomă}
\titlulobtinut{Inginer}
\facultatea{Facultatea de Electronică, Telecomunicații și Tehnologia Informației}
\domeniu{Electronică și Telecomunicații}
\program{ Tehnologii și Sisteme de Telecomunicații}
\director{Conf.Dr.Ing. Ionuţ PIRNOG}
\submissionmonth{Iulie}
\submissionyear{2022}

\begin{document}
%urmatoarea linie inserează paginile obligatorii cf. regulamentului ETTI
\beforepreface{}
\listoffigures
\listoftables
\abbreviations{
	CNN:\ Convolutional\ Neural\ Network
}
\afterpreface{}


%de aici începe teza propriu-zisă
\chapter{Introducere}
Semnele de circulație joacă un rol vital în menținerea siguranței rutiere și a fluidității traficului.
Detectarea și identificarea acestor semne poate fi o sarcină dificilă și de multe ori costisitoare,
deoarece necesită o analiză vizuală atentă a imaginilor și secvențelor video.
În această lucrare de diploma, se va propune o aplicație de detecție și identificare a semnelor de circulație utilizând diverse librării și algoritmi dedicați prelucrării imaginilor/video, cum ar fi OpenCV, YOLOv4, Pytorch, TensorFlow și Python Tesseract. Această aplicație va permite o detectare precisă și rapidă a semnelor de circulație, îmbunătățind astfel siguranța rutieră și eficiența traficului.

Contribuția personală a acestui proiect constă în implementarea și optimizarea unui sistem de recunoaștere a semnelor de circulație în imagini și secvențe video. Proiectul va fi implementat în Python, utilizând diferite librării și algoritmi specifici de prelucrare a imaginilor și algoritmici de machine learning. Acest sistem va fi capabil să detecteze și să identifice semnele de circulație cu o precizie ridicată, prin utilizarea unor modele de învățare profundă, cum ar fi rețele neuronale convoluționale (CNN). În plus, aplicația va fi optimizată pentru a asigura o viteză de procesare ridicată, ceea ce va permite utilizarea sa în timp real în diferite situații de trafic.

În concluzie, această lucrare de diplomă va prezenta o aplicație inovatoare de detecție și identificare a semnelor de circulație, care va îmbunătăți siguranța rutieră și eficiența traficului. Prin implementarea și optimizarea unui sistem de recunoaștere a semnelor de circulație în imagini și secvențe video, acest proiect va reprezenta o contribuție semnificativă la domeniul prelucrării imaginilor și al recunoașterii de modele.
% \bibliographystyle{ieeetr}
% \bibliography{referinte}
\appendix
\chapter{Cod sursă}
\lstinputlisting[caption={Import-uri},linerange={1-20}]{Cod/GTSRB/GTSRB.py}
\lstinputlisting[caption={Definiții de constante si path-uri},linerange={22-39}]{Cod/GTSRB/GTSRB.py}
\lstinputlisting[caption={Funcția care încarcă datele},linerange={41-84}]{Cod/GTSRB/GTSRB.py}
% \lstinputlisting[caption={Import-uri},linerange={1-20}]{Cod/GTSRB/GTSRB.py}
% \lstinputlisting[caption={Import-uri},linerange={1-20}]{Cod/GTSRB/GTSRB.py}
% \lstinputlisting[caption={Import-uri},linerange={1-20}]{Cod/GTSRB/GTSRB.py}
\end{document}
