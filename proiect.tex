\documentclass[12pt, a4paper, twoside, romanian]{teza-etti}
\setcounter{secnumdepth}{3}
\setcounter{tocdepth}{3}
\usepackage{array}
\usepackage{babel}
\usepackage{color}
\usepackage{graphicx}
\usepackage{parskip}
\usepackage{ifthen}
\usepackage{listings}
\usepackage{color}
\usepackage{pgffor}
\usepackage{catchfile}
\usepackage[
	bookmarks,
	bookmarksopen=true,
	pdftitle={Teza},
	linktocpage]{hyperref}

\author{Popescu Ervin-Adrian}
\title{Aplicație de detecție și identificare a semnelor de circulație}
\tiplucrare{Diplomă}
\titlulobtinut{Inginer}
\facultatea{Facultatea de Electronică, Telecomunicații și Tehnologia Informației}
\domeniu{Electronică și Telecomunicații}
\program{ Tehnologii și Sisteme de Telecomunicații}
\director{Conf.Dr.Ing. Ionuţ PIRNOG}
\submissionmonth{Iulie}
\submissionyear{2022}

% listings

\newcounter{FileLines}
\newboolean{RestFile}
\newcommand{\FileLine}{}
\newread\File{}

\newcommand{\CountLinesInFile}[1]
{
	\setboolean{RestFile}{true}
	\setcounter{FileLines}{0}

	\openin\File=#1
	\whiledo{\boolean{RestFile}}
	{
		\ReadNextLine{\File}
		\ifthenelse{\boolean{RestFile}}{
			\stepcounter{FileLines}
		}{}
	}
	\closein\File{}
}

\newcommand{\ReadNextLine}[1]{
	\ifthenelse{\boolean{RestFile}}{
		\read#1 to \FileLine{}
		\ifeof#1\setboolean{RestFile}{false}
		\else % if last line already is read, EOF appears here
		\fi
	}{}
}

\definecolor{codegreen}{rgb}{0,0.6,0}
\definecolor{codegray}{rgb}{0.5,0.5,0.5}
\definecolor{codepurple}{rgb}{0.58,0,0.82}
\definecolor{backcolour}{rgb}{0.95,0.95,0.92}
\lstdefinestyle{mystyle}{
	language=Python,
	backgroundcolor=\color{backcolour},
	commentstyle=\color{codegreen},
	keywordstyle=\color{blue},
	numberstyle=\tiny\color{codegray},
	stringstyle=\color{codepurple},
	basicstyle=\ttfamily\scriptsize,
	breakatwhitespace=false,
	breaklines=true,
	captionpos=b,
	keepspaces=true,
	numbers=left,
	numbersep=5pt,
	showspaces=false,
	showstringspaces=false,
	showtabs=true,
	tabsize=2
}
\lstset{style=mystyle}
\renewcommand{\lstlistingname}{}
\renewcommand{\lstlistlistingname}{List of \lstlistingnames}

% set unnumbered section toc level to 0	
\makeatletter
\providecommand*{\toclevel@unnumberedsection}{0}
\makeatother

\immediate\write18{./Cod/GTSRB/scripts/appendix.py}
% \CatchFileDef{\LineCount}{|"wc -l ./Cod/GTSRB/main.py | awk '{print $1}'"}{}

\begin{document}
%urmatoarea linie inserează paginile obligatorii cf. regulamentului ETTI
\beforepreface{}
\listoffigures
\listoftables
\abbreviations{
	CNN:\ Convolutional\ Neural\ Network
}
\afterpreface{}

%de aici începe teza propriu-zisă
\chapter{Introducere}
Semnele de circulație joacă un rol vital în menținerea siguranței rutiere și a fluidității traficului.
Detectarea și identificarea acestor semne poate fi o sarcină dificilă și de multe ori costisitoare,
deoarece necesită o analiză vizuală atentă a imaginilor și secvențelor video.
În această lucrare de diploma, se va propune o aplicație de detecție și identificare a semnelor de circulație utilizând diverse librării și algoritmi dedicați prelucrării imaginilor/video, cum ar fi OpenCV, YOLOv4, Pytorch, TensorFlow și Python Tesseract. Această aplicație va permite o detectare precisă și rapidă a semnelor de circulație, îmbunătățind astfel siguranța rutieră și eficiența traficului.

Contribuția personală a acestui proiect constă în implementarea și optimizarea unui sistem de recunoaștere a semnelor de circulație în imagini și secvențe video. Proiectul va fi implementat în Python, utilizând diferite librării și algoritmi specifici de prelucrare a imaginilor și algoritmici de machine learning. Acest sistem va fi capabil să detecteze și să identifice semnele de circulație cu o precizie ridicată, prin utilizarea unor modele de învățare profundă, cum ar fi rețele neuronale convoluționale (CNN). În plus, aplicația va fi optimizată pentru a asigura o viteză de procesare ridicată, ceea ce va permite utilizarea sa în timp real în diferite situații de trafic.

În concluzie, această lucrare de diplomă va prezenta o aplicație inovatoare de detecție și identificare a semnelor de circulație, care va îmbunătăți siguranța rutieră și eficiența traficului. Prin implementarea și optimizarea unui sistem de recunoaștere a semnelor de circulație în imagini și secvențe video, acest proiect va reprezenta o contribuție semnificativă la domeniul prelucrării imaginilor și al recunoașterii de modele.

\chapter{Descrierea aplicației}
În {\ref{main.py}} avem \arabic{FileLines}.
\appendix
\chapter{Cod sursă}
\CountLinesInFile{./Cod/GTSRB/main.py}
\lstinputlisting[caption={Main module},linerange={1-\arabic{FileLines}},label=main.py,captionpos=t]{./Cod/GTSRB/main.py}
\pagebreak
\CountLinesInFile{./Cod/GTSRB/modules/config.py}
\lstinputlisting[caption={Config module},linerange={1-\arabic{FileLines}},label=config.py,captionpos=t]{./Cod/GTSRB/modules/config.py}
\pagebreak
\CountLinesInFile{./Cod/GTSRB/modules/custom_model.py}
\lstinputlisting[caption={Custom model module},linerange={1-\arabic{FileLines}},label=custom_model.py,captionpos=t]{./Cod/GTSRB/modules/custom_model.py}
\pagebreak
\CountLinesInFile{./Cod/GTSRB/modules/load_data.py}
\lstinputlisting[caption={Load data module},linerange={1-\arabic{FileLines}},label=load_data.py,captionpos=t]{./Cod/GTSRB/modules/load_data.py}
\pagebreak
\CountLinesInFile{./Cod/GTSRB/modules/logger.py}
\lstinputlisting[caption={Logger module},linerange={1-\arabic{FileLines}},label=logger.py,captionpos=t]{./Cod/GTSRB/modules/logger.py}
\pagebreak
\CountLinesInFile{./Cod/GTSRB/modules/videowriter.py}
\lstinputlisting[caption={Videowriter module},linerange={1-\arabic{FileLines}},label=videowriter.py,captionpos=t]{./Cod/GTSRB/modules/videowriter.py}
\pagebreak

% \bibliographystyle{ieeetr}
% \bibliography{referinte}
\end{document}

